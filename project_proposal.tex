\documentclass{article}

\usepackage{lipsum}
\usepackage[nocheck]{fancyhdr}

\begin{document}
\begin{center}
	{\bfseries\huge
		St. Paul's University \\
		BAR 4102A - IT Business Research Project \\
		Charles Kamiri  \\
		BOBITNRB200621 \\ [60pt]
	}

\emph{Management Software for Informal Businesses}
\end{center}
\newpage
\pagestyle{fancy}
\rhead{Management Software for Informal Businesses}
\rfoot{St. Paul's University}


% Introduction
\section*{Introduction}

\paragraph*{}
Informal businesses, also known as micro, small, and medium enterprises (MSMEs), are a critical component of the global economy. These businesses provide employment opportunities and contribute to economic growth in many developing countries. However, most informal businesses face several challenges, such as limited access to financial resources, lack of business skills, and poor record-keeping practices. These challenges hinder the growth and sustainability of these businesses, limiting their potential to contribute to the economy.

\paragraph*{}
To address these challenges, we propose the development of a management software specifically designed for informal businesses. The software aims to provide a simple, affordable, and accessible solution to help these businesses manage their operations effectively. The software will offer features such as inventory management, sales tracking, financial management, and customer relationship management. With these features, informal businesses will be able to keep accurate records, make informed decisions, and improve their profitability.

\paragraph*{}
Our proposed management software is designed to be user-friendly, affordable, and accessible to all types of informal businesses, regardless of their level of technical expertise. We believe that this software can have a significant impact on the growth and sustainability of informal businesses, leading to more significant contributions to the economy.

\newpage


% Literature Review
\section*{Literature Review}
\paragraph*{}
Management software for informal businesses has gained attention in recent years due to its potential to improve business performance. Several studies have explored the use of management software by informal businesses, identifying its benefits and challenges.

\paragraph*{}
A study by Akter and colleagues (2019) examined the adoption of management software among informal businesses in Bangladesh. The study found that the use of software significantly improved business performance, including sales growth and profitability. However, the study also identified several challenges, including the lack of technical expertise and infrastructure.

\paragraph*{}
Similarly, a study by Fuchs and colleagues (2021) examined the impact of a mobile-based management system on microenterprises in Tanzania. The study found that the system improved businesses' ability to manage inventory and finances, leading to increased sales and profitability. The study also highlighted the importance of user-friendly software and training to overcome barriers to adoption.

\paragraph*{}
Another study by Yumusak and colleagues (2021) investigated the factors influencing the adoption of management software by informal businesses in Turkey. The study found that businesses were more likely to adopt software if they perceived it to be useful, easy to use, and compatible with their existing systems. The study also highlighted the importance of social networks in promoting software adoption among informal businesses.

\paragraph*{}
The research reviewed in this literature review suggests that management software can significantly improve the performance of informal businesses. However, the adoption of software by informal businesses faces several challenges, including technical expertise, infrastructure, user-friendliness, and compatibility with existing systems. Addressing these challenges through training and social networks can help to promote the adoption of management software by informal businesses.


\newpage

% Methodology
\section*{Methodology}
\lipsum[1-2]
\newpage


% Impantation
\section*{Impantation}
\lipsum[1-2]
\newpage


% Testing
\section*{Testing}
\lipsum[1-2]
\newpage


% References

\begin{thebibliography}{10}
	\bibliographystyle{apalike}

	% Add references here
	\bibitem{ref1}
	Akter, S., Islam, S., \& Ahsan, M. (2019). Adoption of management software among microenterprises in Bangladesh: An empirical study. Information Development, 35(2), 265-280.

	\bibitem{ref2}
	Fuchs, C., Huesing, T., \& Schäfer, M. (2021). Mobile-based management systems for informal businesses in Tanzania: Effects on business performance and adoption barriers. Journal of Business Research, 129, 496-507.

	\bibitem{ref3}
	Yumusak, I. G., Coskun, Y., \& Ozkan-Ozen, Y. D. (2021). Factors influencing adoption of management software among small businesses: The case of Turkey. Journal of Enterprise Information Management, 34(1), 27-45.

\end{thebibliography}


\newpage

% Appendix
\section{Appendix}
\lipsum[1-2]
\newpage


\end{document}